\chapter{\uppercase{Conclusions and Future Enhancements}}
The conclusion will need to have several elements, including:

\begin{itemize}
    \item A brief summary, just a few paragraphs, of your key findings, related back to what you expected to see (essential);
    \item The conclusions which you have drawn from your research (essential);
    \item Why your research is important for researchers and practitioners (essential);
    \item Recommendations for future research (strongly recommended, verging on essential);
    \item Recommendations for practitioners (strongly recommended in management and business courses and some other areas, so check with your supervisor whether this will be expected); and a final paragraph rounding off your dissertation or thesis.
\end{itemize}

In any research-oriented work, references and citations play a crucial role in establishing the credibility and reliability of the study. Citations are used within the main body of the report to acknowledge the original sources of ideas, theories, methodologies, and results that have informed the research. They allow the reader to trace the origin of information, verify claims, and gain deeper insight into the background literature\cite{waldron2003generalized}.

References, on the other hand, are listed at the end of the dissertation or thesis. They provide the complete bibliographic details of all the sources cited in the text. A well-structured reference list demonstrates the breadth and depth of the literature consulted during the research process. It also helps future researchers to locate relevant works more efficiently\cite{kothari2004research}.

In this report, references have been cited in accordance with standard academic practice. Each in-text citation corresponds to a detailed entry in the reference list. Sources include journal papers, conference proceedings, books, technical reports, and reputable online resources. The citation style followed (e.g., IEEE, APA, or any style recommended by the institute) ensures uniformity and consistency throughout the document\cite{haykin2005cognitive}.
