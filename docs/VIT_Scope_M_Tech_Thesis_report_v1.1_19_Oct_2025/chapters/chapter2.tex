\chapter{\uppercase{Literature Review}}
\section{Review of Literature}
A literature review is a text of a scholarly paper, which contains the current knowledge 
including substantive findings, as well as theoretical and methodological contributions to 
a particular topic. Literature reviews are secondary sources, and do not report new or 
original experimental work. Most often associated with academic-oriented literature, 
such reviews are found in academic journals, and are not to be confused with book reviews 
that may also appear in the same publication. Literature reviews are a basis for research 
in nearly every academic field (Waldron, 2008b). 

A narrow-scope literature review may be included as part of a peer-reviewed journal article 
presenting new research, serving to situate the current study within the body of the relevant 
literature and to provide context for the reader. In such a case, the review usually precedes 
the methodology and results sections of the work. Producing a literature review may also be 
part of graduate and post-graduate student work, including in the preparation of a thesis, 
dissertation, or a journal article. Literature reviews are also common in a research proposal 
or prospectus (the document that is approved before a student formally begins a dissertation 
or thesis) (Doan et al., 2002; Duzdevich et al., 2014; Ganesh et al., 2016).

\section{Review Types}
The main types of literature reviews are: evaluative, exploratory, and instrumental 
(Duzdevich et al., 2014; Neil and David, 2016). 

A fourth type, the systematic review, is often classified separately, but is essentially a 
literature review focused on a research question, trying to identify, appraise, select and 
synthesize all high-quality research evidence and arguments relevant to that question. 
A meta-analysis is typically a systematic review using statistical methods to effectively 
combine the data used on all selected studies to produce a more reliable result \cite{duzdevich2014dna}, \cite{mohanaprasad2017optimized}, \cite{gilbarg1977elliptic}, \cite{haykin2004kalman}, \cite{haykin2005cognitive}, \cite{knuth1986computers}.

\subsection{Process and Product}
Gilbarg and Trudinger (2015) distinguish between the process of reviewing the literature 
and a product known as a literature review. The process of reviewing the literature is 
often ongoing and informs many aspects of the empirical research project.

\section{Page Limit of Review}
A careful literature review is usually between 8 to 20 pages. The process of reviewing 
the literature requires different kinds of activities and ways of thinking. 
(Duzdevich et al., 2014) and (Kothari, 2004) link the activities of doing a literature review 
with Benjamin Bloom’s revised taxonomy of the cognitive domain (ways of thinking, 
remembering, understanding, applying, analysing, evaluating, and creating).